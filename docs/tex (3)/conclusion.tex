% !TeX spellcheck = ru_RU
\chapter*{\hfill{\centering ЗАКЛЮЧЕНИЕ}\hfill}
\addcontentsline{toc}{chapter}{ЗАКЛЮЧЕНИЕ}

В рамках данной научно-исследовательской работы был проведен анализ методов оптимизации для решения задачи упаковки контейнеров в трюмы кораблей.

Достигнута поставленная цель. Были выполнены следующие задачи:
\begin{itemize}
	\item проведен анализ предметной области;
	\item выбраны и формализованы критерии и ограничения оптимизации;
	\item формализована и разработана целевая функция;
	\item проведен анализ эвристических методов оптимизации.  
\end{itemize}

В результате сравнения и классификации эвристических методов оптимизации для решения поставленной задачи был выбран генетический алгоритм. Данный метод удовлетворяет всем необходимым критериям. Алгоритм обладает сходимостью за полиномиальное время и позволяет параллельно обрабатывать данные - что необходимо для решения поставленной задачи, требующей большого объема вычислений.
Данный метод устойчив к локальным оптимумам, что крайне важно для оптимизационных задач. Генетический алгоритм также обладает гибкостью и интуитивно понятной настройкой параметров, поэтому является наиболее подходящим методом для решения задачи упаковки контейнеров в трюм корабля.