% !TeX spellcheck = ru_RU
%\chapter*{ВВЕДЕНИЕ}
\chapter*{\hfill{\centering ВВЕДЕНИЕ}\hfill}
\addcontentsline{toc}{chapter}{\normalsize\bfseries\centering ВВЕДЕНИЕ}

В условиях глобализации и быстрого роста международной торговли объем морских грузоперевозок постоянно увеличивается. Это делает задачу оптимизации логистических процессов и транспортировки грузов особенно актуальной для судоходных компаний, логистических операторов и производителей. Эффективное управление грузопотоками, рациональное использование транспортных ресурсов и оптимальная загрузка трюмов судов позволяют снизить затраты на перевозки, повысить производительность логистических операций и минимизировать воздействие на окружающую среду. При решении данной задачи важно учитывать размеры контейнеров и трюмов судов.

Целью данной работы является анализ методов оптимизации для задач упаковки контейнеров в трюмы кораблей.

Достижение поставленной цели требует решения следующих задач:
\begin{itemize}
	\item провести анализ предметной области;
	\item выбрать и формализовать критерии и ограничения оптимизации;
	\item формализовать и разработать целевую функцию;
	\item провести анализ эвристических методов оптимизации.  
\end{itemize}