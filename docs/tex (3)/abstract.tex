% !TeX spellcheck = ru_RU

%\chapter*{РЕФЕРАТ}

\chapter*{\hfill{\centering РЕФЕРАТ}\hfill}
\addcontentsline{toc}{chapter}{РЕФЕРАТ}

Расчетно-пояснительная записка 19 с., 1 табл., 11 ист., 1 прил.

Ключевые слова: ТРЮМ, КОНТЕЙНЕР, ЗАДАЧА ТРЕХМЕРНОЙ УПАКОВКИ, ОПТИМИЗАЦИЯ, ЭВРИСТИЧЕСКИЕ АЛГОРИТМЫ, ГЕНЕТИЧЕСКИЙ АЛГОРИТМ.

Целью данной выпускной квалификационной работы является постановка и формализация задачи упаковки контейнеров в трюмы судна, выбор ограничений и критериев оптимизации, анализ эвристических методов оптимизации и выбор метода для решения поставленной задачи.

Объектом исследования в данной работе является процесс упаковки контейнеров в трюм корабля. Для оптимизации данного процесса необходимо выделить ограничения и критерии, а также построить математическую модель. Проблема упаковки контейнеров в трюм корабля перекликается с задачей трехмерной упаковки блоков, которую не решить точными методами. Поэтому для решения поставленной задачи рассматриваются и классифицируются эвристические алгоритмы.